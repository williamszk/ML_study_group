





\documentclass{beamer}
\renewcommand{\baselinestretch}{1.2}

\usepackage[english]{babel}
\usepackage[utf8]{inputenc}

\usepackage{subfig}
\usefonttheme{serif}
\usepackage{amsmath}
\usepackage{amsfonts}
\usepackage{mathtools}
%\usepackage{bbm}
%\usepackage{float}
%\usepackage{array}
\usepackage{geometry}
\usepackage{amsfonts}
\usepackage{microtype}
\usepackage{accents}
%\usepackage{relsize}
\usepackage{bm}
\usepackage{caption}

\usepackage{lscape}
\usepackage{pdflscape}
\usepackage{rotating}
\usepackage{grffile}
\usepackage{url}
\usepackage{adjustbox}
\usepackage{multirow}
\usepackage{tabularx, blkarray}
\usepackage{eqparbox} 
\usepackage{comment}
\newcommand{\Lagr}{\mathcal{L}}
\usepackage{listings}


\usepackage{amssymb}
\newcommand{\E}{\mathrm{E}}
\newcommand{\Var}{\mathrm{Var}}
\newcommand{\Cov}{\mathrm{Cov}}
\renewcommand{\P }{\mathrm{P }}
\renewcommand{\ln }{\mathrm{ln }}

\usepackage{natbib}
\bibliographystyle{chicago}





 
 

\title{ An Introduction to \\ Tree-Based Methods}
\author{\small William  Suzuki}
\institute{FEARP-USP}
\date{2019}
 
 
\begin{document}
 
\frame{\titlepage}
 



\begin{frame}
\frametitle{Regression Trees}
Types of methods: Regression Trees, Bagging, Random Forests, Boosting.
\\~\\
Regression Trees are simple and useful for interpretation. 

Problem: poor prediction. 
Typically worse than: LASSO, Ridge, Splines, Generalized Additive Models (GAM).
\\~\\
\textit{Bagging, Random Forests, Boosting}: combine a large number of trees. Improve prediction accuracy. 

Problem: at the expense of some loss in interpretation.
\end{frame} 





\begin{frame}
\frametitle{Regression Trees}
Source: \textit{Gareth James, Daniela Witten, Trevor Hastie, Robert Tibshirani;} An Introduction to Statistical Learning: with Applications in R
\\~\\

Motivation: Predicting Baseball Players’ Salaries Using Regression Trees
\\~\\
$ y= \{ \text{log salary of a baseball player}  \} $

$ X = \{ \text{years playing;  last year's number of hits} \}  $
\end{frame} 

\begin{frame}
\frametitle{Regression Trees}
\begin{figure}[H]
\caption{\texttt{Hitters} regression tree example}
\includegraphics[width=5cm,height=5cm]{"hitters1".JPG} 
\end{figure}

$ y= \{ \texttt{Salary}  \} $

$ X = \{ \texttt{Years, Hits} \}  $
\end{frame} 



\begin{frame}
\frametitle{Methodology}

We define the following regions for the covariates $X$:

$ R_1 = \{X| \texttt{Years<4.5}  \}$ 

$ R_2 = \{X| \texttt{Years>=4.5, Hits<117.5} \}$  

$ R_3 = \{X| \texttt{Years>=4.5, Hits>=117.5} \} $
\\~\\
$ R_1, R_2, R_3 $ are called \textit{terminal nodes} or \textit{leaves}.
\\~\\
The points along the tree where the predictor space is split are referred to as \textit{internal nodes}. 
\end{frame} 




\begin{frame}
\frametitle{Regression Trees}
\begin{figure}[H]
\caption{\texttt{Hitters} three-region partition}
\includegraphics[width=6cm,height=5cm]{"hitter2".JPG} 
\end{figure}
\end{frame} 




\begin{frame}
\frametitle{Regression Trees}

Interpretation: 
\\~\\
\texttt{Years} is the most important factor in determining  \texttt{Salary}. Less experience means lower salary.

\texttt{Hits} are not model relevant for lesser experienced players. 
\\~\\
\texttt{Hits} are model relevant for more experienced players (5+ years). 

More \texttt{Hits} means higher \texttt{Salary}.
\\~\\
Good: easy to interpret, and nice graphical representation.

Problem: model too simple.

\end{frame} 



\begin{frame}
\frametitle{Regression Trees}
\begin{figure}[H]
\caption{Visualize Regression Tree}
\includegraphics[width=8.5cm,height=7cm]{"tree1".JPG} 
\end{figure}
\end{frame} 







\begin{frame}
\frametitle{Regression Trees}

Algorithm steps:
\\~\\
For any $j$ and $s$, we define the pair of half-planes:
$$ R_1 (j,s) = \{ X|X_j < s \}  \text{ and }  R_2 (j,s) = \{ X|X_j \geq s \}  $$
we seek the value of $j$ and $s$ that minimize the equation
 $$ \min_{j,s}   \sum\limits_{ i \text{ in } R_1 } ( y_i - y_{R_1}  )^2 + \sum\limits_{ i \text{ in } R_2 } ( y_i - y_{R_2}  )^2  $$



\end{frame} 






\begin{frame}
\frametitle{Regression Trees}

\textbf{Tree Pruning} 
\\~\\
Problem: The algorithm goes till a stop criterion, the resulting tree might be too complex. Overfitting.
\\~\\
Strategy: grow a very large tree $T_0$ and then prune $T_0$ in order to obtain a \textit{subtree}.

\end{frame} 




\begin{frame}
\frametitle{Regression Trees}

\textbf{Weakest Link Pruning} 
\\~\\
$ \alpha $ is a nonnegative tuning parameter.


For each subtree $T  \subset T_0 $
$$ \sum\limits_{m=1}^{|T|}  \left[ \sum\limits_{i \text{ in } R_m  } ( y_i - y_{R_m}  )^2   \right]  + \alpha |T| $$
$|T|$ number of leafs in the tree.

Use cross-validation to select $ \alpha $.


\end{frame} 



\begin{frame}
\frametitle{Regression Trees}
\begin{figure}[H]
\caption{Unpruned Regression Tree for \texttt{Hitters}}
\includegraphics[width=8.5cm,height=7cm]{"tree2".JPG} 
\end{figure}
\end{frame} 


\begin{frame}
\frametitle{Regression Trees}
\begin{figure}[H]
\caption{MSE graphs for \texttt{Hitters}}
\includegraphics[width=9.5cm,height=7cm]{"graph1".JPG} 
\end{figure}
\end{frame} 


\begin{frame}
\frametitle{Classification Trees}

A classification tree is very similar to a regression tree, except that it is tree used to predict a qualitative response rather than a quantitative one. 
\\~\\
In classification trees we predict that each observation belongs to the most commonly occurring class of training observations in the region to which it belongs.
$$ \forall m \quad\quad\quad\quad 	G_m = \sum\limits_{k=1}^{K} p_k (1-p_k)  \quad\quad\quad\quad  D_m =  \sum\limits_{k=1}^{K} p_k \log p_k $$

\end{frame} 



\begin{frame}
\frametitle{Regression Trees}

\begin{itemize}
\item Trees are very easy to explain to people.
\item Trees can be displayed graphically, and are easily interpreted.
\item Trees can easily handle qualitative predictors.
\item (bad) Comparatively low predictive power.
\end{itemize}
Next: Bagging, Random Forests, Boosting. 
\end{frame} 



\begin{frame}
\frametitle{Bagging}


\textit{Bootstrap aggregation} $=$ \textit{bagging}  is a general-purpose procedure for reducing the bagging variance of a statistical learning method.
\\~\\
Bagging typically results in improved accuracy over prediction using a single tree. Unfortunately, however, it can be difficult to interpret the resulting model.
\\~\\
The bagging estimator is:
$$ \hat{f}(x) = \frac{1}{B} \sum\limits_{b=1}^{B} \hat{f}_b(x)  $$

\end{frame} 



\begin{frame}
\frametitle{Out-of-Bag}

out-of-bag (OOB) 

One can show that on average, each bagged tree makes use of around two-thirds of the observations. 

This will yield around B/3 predictions for the ith observation. 

OOB MSE (for a regression problem) or classification error (for a
classification problem) 

It can be shown that with B sufficiently large, OOB error is virtually equivalent to leave-one-out cross-validation error.


\end{frame} 


\begin{frame}
\frametitle{Methodology}


Random forests provide an improvement over bagged trees by way of a
random
forest small tweak that decorrelates the trees. As in bagging, we build a number
of decision trees on bootstrapped training samples.

each time a split in a tree is considered, a random sample of $m$ predictors is chosen as split candidates from the full set of $p$ predictors.

$$ m  \approx \sqrt{p} $$

\end{frame} 



\begin{frame}
\frametitle{Methodology}


This may sound crazy, but it has a clever rationale. Suppose
that there is one very strong predictor in the data set, along with a num-
ber of other moderately strong predictors. Then in the collection of bagged
trees, most or all of the trees will use this strong predictor in the top split.
Consequently, all of the bagged trees will look quite similar to each other.Hence the predictions from the bagged trees will be highly correlated.Un-
fortunately, averaging many highly correlated quantities does not lead to
as large of a reduction in variance as averaging many uncorrelated quanti-
ties.
\\~\\
We can think of this process as decorrelating
the trees, thereby making the average of the resulting trees less variable
and hence more reliable
\\~\\
if $m=p$ then random forest is the same as bagging



\end{frame} 


\begin{frame}
\frametitle{Regression Trees}
\begin{figure}[H]
\caption{Classification error graphs for \texttt{Heart}}
\includegraphics[width=9.5cm,height=7cm]{"bag1".JPG} 
\end{figure}
\end{frame} 



\begin{frame}
\frametitle{Regression Trees}
\begin{figure}[H]
\caption{Classification error graphs for \texttt{Heart}}
\includegraphics[width=9.5cm,height=7cm]{"genes1".JPG} 
\end{figure}
\end{frame} 

\begin{frame}
\frametitle{Boosting}

Algorithm steps: Boosting for Regression Trees

\begin{enumerate}
\item  Set $f(x) = 0$ and $r = y$ for all $i$ in the training set.
\item For $b = 1,2,...,B$, repeat: 
\begin{enumerate}
\item Fit a tree $f_b$ with $d$ splits ($d+1$ leafs) to the training data $(X,r)$.
\item Update $f$ by adding in a shrunken version of the new tree: 
$$  f(x) \leftarrow f(x) + \lambda f_b(x) $$
\item Update the residuals,
$$ r \leftarrow r  - \lambda f_b (x) $$
\end{enumerate}
\item  Output the boosted model, 
$$ f(x) = \sum\limits_{b=1}^{B}  \lambda f_b (x) $$
\end{enumerate}

\end{frame} 



\begin{frame}
\frametitle{Boosting}

Usually $\lambda = 0.01 \text{ or } 0.001$





\end{frame} 





\begin{frame}
\frametitle{Methodology}







\end{frame} 









\begin{frame}
\frametitle{References}
\begin{tiny}
\bibliography{references}
\end{tiny}
\end{frame} 






















\end{document}

 
